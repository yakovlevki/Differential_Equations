% Л.С. Понтрягин, ОДУ
% В.И. Арнольд, ОДУ
% А.Ф. Филипов, Л.ОДУ
\section{Лекция 1}
\begin{definition}
    Дифференциальным уравнением первого порядка в нормальной форме называется уравнение 
    \[\dot{x}=v(t,x)\]
    где $x\in M,\ t\in \R_t$.
\end{definition} 
\begin{definition}
    Область опредения дифференциального уравнения - это область, где определено $v$.
\end{definition} 
\begin{definition}
    $x$ называется точкой фазового пространства, $(t,x)$ называется точкой расширенного фазового пространства.
\end{definition} 
\begin{example} (Модель Мальтуса)\\
    Гипотеза: масса популяции "растет" со скоростью, пропроциональной имеющейся массе. Пусть $m$ - масса популяции, $r$ - коэффициэнт пропорциональности. При первом приближении обычно полагают $r:= b-d$, где $b$ - коэффициэнт рождаемости, $d$ - коэффициэнт смертности. Тогда модель популяции описывается уравнением
    \[\dot{m}=rm\]
    перейдем к дифференциалам
    \[\dot{m}dt=rmdt\]
    \[\frac{dmt}{mt}=rdt\]
    проинтегрируем обе части
    \[\int\limits_{t_0}^{t}\frac{dm}{m}=\int\limits_{t_0}^{t}rdt\]
    \[\ln{\left|\frac{m(t)}{m(t_0)}\right|=r(t-t_0)}\]
    отсюда
    \[\boxed{m(t)=m(t_0)e^{r(t-t_0)}}\]
\end{example}
\begin{example} (Модель Ферхюльста)\\
    Положим $r=r(m)=a-bm$
    \[\dot{m}=(a-bm)m\]
    \[\dot{m}=r(1-\frac{m}{k})m\]
\end{example}
\begin{example} (Модель Ланкастера)\\
    Пусть $x$ - одна популяция, $y$ - другая популяция
    \[\begin{cases}
        \dot{x}=-ay,\ a>0,\\
        \dot{y}=-bx,\ b>0.
    \end{cases}\]
    домножим первое уравнение на $bxdt$, а второе на $aydt$ и вычтем второе из первого:
    \[bx\dot{x}dt-ay\dot{y}dt=0\]
    \[bxdx-aydy=0\]
    проинтегрировав, получим
    \[\boxed{x^2-ay^2=c}\]
\end{example}
%\begin{example}
%    Обмен знаниями (технологиями, энергией)
%\end{example}
\begin{definition}
    Векторное поле $v$ называется автономным, если оно не зависит от времени, неавтономным в противном случае.
\end{definition} 
\begin{definition}
    Решением дифференциального уравнения
    \[\dot{x}=v(t,x)\]
    с областью определения $D\subset \R_t\times M$ на промежутке $I$ называется дифференцируемым отображением $t \to x(t),\ t\in I$, такое, что
    \begin{enumerate}
        \item $(t,x(t))\in D,\ t\in I$
        \item $\dot{x}\equiv v(t, x(t)),\ t\in I$.
    \end{enumerate}
\end{definition} 
\begin{definition}
    Отображение $(M_1,\rho_1) \to (M_2, \rho_2)$ удовлетворяет условию Липшица, если существует $L>0$ такая, что
    \[\rho_2(f(x_1),f(x_2))\leq L\cdot\rho_1(x_1,x_2)\]
\end{definition} 
% Упражнение: дифф => липш => непрер (локально) и что обратные стрелки неверны
\begin{definition}
    Задачей Коши называется поиск решения, проходящего через заданную точку области опредения.
\end{definition} 
\begin{theorem} (Теорема существования и единственности решения задачи Коши)\\
    Если в области $D$ векторное поле $v$ непрерывно по $t$ и $x$ и локально вблизи каждой точки удовлетворяет условию Липшица по $x$, то решение задачи Коши 
    \[\dot{x}=v(t,x),\ x(t_0)=x_0,\ (t_0, x_0)\in D\]
    существует и единственно при временах достаточно близких к $t_0$. %тут еще чето хз
\end{theorem} 
\begin{exercise} Убедиться, что при $r>0$ в моделе
    \[\dot{m}=rm^2\]
    популяция стремится к бесконечности за конечное время
\end{exercise}
\begin{exercise} Убедиться, что решение уравнения
    \[y'=y^{\frac{2}{3}}\]
    имеет вид кубической параболы.
\end{exercise}